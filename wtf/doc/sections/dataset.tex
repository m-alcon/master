\section{Data Set}
Our work is based on the processed Cleveland data offered by the Hearth Disease data set. The Cleveland data is composed of 14 features of the patient and its heart, which are:
\begin{itemize}
    \item \textbf{Age.} In years.
    \item \textbf{Sex.} Binary representation of the patient's gender.
    \item \textbf{Chest pain type.} Integer value between 0 and 3, representing if the pain is a typical angina, atypical angina, non-anginal pain or asymptomatic.
    \item \textbf{Resting blood pressure}. In mm Hg on admission to the hospital.
    \item \textbf{Serum cholesterol.} In mg/dl.
    \item \textbf{Fasting blood sugar.} Binary representation of: the fasting blood sugar of the patient $>$ 120 mg/dl.
    \item \textbf{Resting electrocardiographic results.} Integer values between 0 and 2, representing the severity of the results.
    \item \textbf{Maximum heart rate achieved.} Integer value.
    \item \textbf{Exercise induced angina.} Binary representation of `yes' or `no'.
    \item \textbf{Old peak.} Decimal value that represents the ST depression induced by exercise relative to rest.
    \item \textbf{Slope.} Three integer values representing the slope of the peak exercise ST segment.
    \item \textbf{Number of major vessels colored by fluoroscopy} Integer value between 0 and 3.
    \item \textbf{Thalassemia.} Three integer values representing whether it is a fixed or reversible defect, or the observations are normal.
    \item \textbf{Target}. Integer value between 0 and 4 that represents the presence of hearth disease in the patient.
\end{itemize}
For more information about the features take a look at \cite{dataset}.