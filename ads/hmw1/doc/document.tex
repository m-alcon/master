\documentclass[a4paper, 10pt]{article}
\usepackage{fullpage} % changes the margin
\usepackage[english]{babel}
\usepackage[utf8]{inputenc}
\usepackage{hyperref}
\usepackage{xcolor}
\usepackage{graphicx}
\usepackage{array}
\usepackage{float}
\usepackage{longtable}
\usepackage[bottom]{footmisc}
\usepackage{cite}
\usepackage{parskip}
\usepackage{subcaption}
\usepackage{amssymb}
\usepackage{amsmath}
\usepackage{listings}

\hypersetup{
	colorlinks=true,       % false: boxed links; true: colored links
	linkcolor=blue,        % color of internal links
	citecolor=blue,        % color of links to bibliography
	filecolor=magenta,     % color of file links
	urlcolor=blue
}

%\setlength{\parindent}{0cm}
\newcommand{\code}[1]{\texttt{#1}}
\renewcommand{\arraystretch}{1.4}

\graphicspath{{img/}}

\begin{document}

\noindent
\begin{flushright}
    \large\textbf{Miguel Alcón Doganoc} \\
    Advance Data Structures \\
    \today
\end{flushright}

\noindent
{\huge{\textbf{Vantage Point Trees}}}

\section*{Introduction}
In this document I am going to compare two implementations of the vantage point trees (\textit{VPT}), one written in \textit{Python} \cite{python} and the other one in \textit{C++} \cite{cpp}. With this, we can compare which of the programming languages perform better with this kind of data structure.

\section*{Experiment}
In order to compare both implementations, I designed the following experiment. First, in \textit{C++} I generated randomly a vector of 3-dimensional points of size 10000, with real values between 100 and -100. Then, I created the \textit{VPT} with the vector. Finally, I searched 100 neighbors of a random point, 100 times. I saved the vector of points, the searched points, the results, and the execution times in a file, to reproduce the same experiment with the other implementation and compare the results.

With the \textit{Python} implementation I did the same, but instead of generating the data, I read from the saved files.

\section*{Results}
As a result of the experiment, I saw that the \textit{C++} implementation performs much better than the \textit{Python} one, in both time and found neighbors.

In the case of time, \textit{Python} lasts 21.930 s to construct the tree, 0.0163 s to search 100 neighbors of a point on average, and 0.0234 s at most. In the other hand, \textit{C++} lasts 2.831 s to construct the tree, 0.0004 s to search 100 neighbors of a point on average, and 0.0027 s at most.

In the other case, the \textit{C++} implementation always returns the 100 closest neighbors to the searched point, while in some cases the \textit{Python} implementation returns other ones or even less than 100 neighbors. Moreover, the \textit{C++} implementation has more decimal precision. As a final comment, the \textit{Python} implementation returned the correct neighbor points 97 of the 100 executions.

\bibliographystyle{unsrt}
\bibliography{cite}


\end{document}