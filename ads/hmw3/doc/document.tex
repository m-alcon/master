\documentclass[a4paper, 10pt]{article}
\usepackage{fullpage} % changes the margin
\usepackage[english]{babel}
\usepackage[utf8]{inputenc}
\usepackage{hyperref}
\usepackage{xcolor}
\usepackage{graphicx}
\usepackage{array}
\usepackage{float}
\usepackage{longtable}
\usepackage[bottom]{footmisc}
\usepackage{cite}
\usepackage{parskip}
\usepackage{subcaption}
\usepackage{amssymb}
\usepackage{amsmath}
\usepackage{listings}

\hypersetup{
	colorlinks=true,       % false: boxed links; true: colored links
	linkcolor=blue,        % color of internal links
	citecolor=blue,        % color of links to bibliography
	filecolor=magenta,     % color of file links
	urlcolor=blue
}

%\setlength{\parindent}{0cm}
\newcommand{\code}[1]{\texttt{#1}}
\renewcommand{\arraystretch}{1.4}

\graphicspath{{../plots/}}

\begin{document}

\noindent
\begin{flushright}
    \large\textbf{Miguel Alcón Doganoc} \\
    Advanced Data Structures \\
	%\today
	June 21, 2019
\end{flushright}

\noindent
{\huge{\textbf{Dictionary with Binary Search Trees}}}

\section*{Introduction}
The objective of this work is to implement a dictionary using a binary search tree (BST) as the base of the data structure, and then compare it with the default \textit{dict} class of python. I started my implementation from the \textit{Python} BST implementation in \cite{python}.

\section*{Implementation}
First, I moved the code from \textit{Python 2} to \textit{Python 3} because I am used to work with \textit{Python 3}. After that, I modified the \textit{Node} class to let it accept a \textit{key} and a \textit{value}. The main idea of this modification is to use the \textit{key} as the normal value of a BST, and add a new parameter in the node saving the \textit{value} corresponding to the \textit{key} of the node. In my opinion, this is a simple modification that adds a great functionality to the data structure.

Regarding the operations, all the standard ones were done in the implementation that I took from the Internet, but the deletion operation was not completed, and the insertion operation does not behave as I wanted. The deletion operation did not take into account that a node can have no parent, hence when I wanted to delete the root node of the main tree, it did nothing. For each of the cases (0, 1 and more than 1 children), I implemented the behavior of the deletion operation for the root node of the main tree. For the insertion operation, I had to force the tree to not allow multiple \textit{keys} by modifying the operation itself. Instead of adding a pair \textit{key-value} without checking if the \textit{key} is already in the tree, now the BST dictionary checks it and swaps the previous \textit{value} with the new one, as the \textit{Python} dictionary behaves.

\section*{Experiment}
As it is known, the \textit{Python} dictionary allows the \textit{key} to be whatever hashable class, and the \textit{value} whatever class. In my case, the \textit{key} has to be a comparable and ordered class. With my own implementation I tried to insert different types of \textit{values} (boolean, dictionaries and lists), with different types of \textit{keys} (integer, float, list and dictionaries).

After that, I tested the performance of the main operations of the BST dictionary and the \textit{Python} one. In order to do that, I generated randomly 1000 integer \textit{keys} and another 1000 integer \textit{values}. First, I inserted them to both dictionaries, then I searched each of the keys, and finally I deleted all of them one by one.


\section*{Results}
In the type acceptance test, only when I tried to insert a pair \textit{key-value} with a dictionary as a \textit{key}, the BST crashed. I checked it for the \textit{Python} one and it also failed.

In the performance test, the \textit{Python} dictionary outperforms the BST one, as you can see in table \ref{tab}.

\begin{table}[hbtp]
	\centering
	\begin{tabular}{c c c}
		\textbf{Operation} & \textbf{Python} & \textbf{BST} \\ \hline
		Insertion & 0.3516 & 28.728 \\
		Search & 0.2737 & 4.059 \\
		Deletion & 0.3114 & 3.905 	\\		
	\end{tabular}
	\caption{Performance results (in milliseconds)}
	\label{tab}
\end{table}

\bibliographystyle{unsrt}
\bibliography{cite}


\end{document}