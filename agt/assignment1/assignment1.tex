\documentclass[a4paper, 11pt]{article}
\usepackage{fullpage} % changes the margin
\usepackage[english]{babel}
\usepackage[utf8]{inputenc}
\usepackage{hyperref}
\usepackage{xcolor}
\usepackage{graphicx}
\usepackage{array}
\usepackage{float}
\usepackage{longtable}
\usepackage[bottom]{footmisc}
\usepackage{cite}
\usepackage{parskip}
\usepackage{subcaption}
\usepackage{amssymb}
\usepackage{amsmath}
\usepackage{listings}

\hypersetup{
	colorlinks=true,       % false: boxed links; true: colored links
	linkcolor=blue,        % color of internal links
	citecolor=blue,        % color of links to bibliography
	filecolor=magenta,     % color of file links
	urlcolor=blue
}

%\setlength{\parindent}{0cm}
\newcommand{\code}[1]{\texttt{#1}}
\newcommand{\onemm}[0]{\hspace{1mm}}
\newcommand{\suchthat}[0]{\onemm|\onemm}
\renewcommand{\arraystretch}{1.4}
\definecolor{statement}{gray}{0.5}

\graphicspath{{img/}}

\begin{document}

\noindent
\begin{flushright}
    \large\textbf{Miguel Alcón Doganoc} \\
    Algorithms for Game Theory \\
    \today
\end{flushright}

\noindent
{\huge{\textbf{First Problem Assignment}}}

\section*{Problem 6}
{\color{statement}
\subsection*{Statement}
Consider a set of $n$ players that must be partitioned into two groups. However, there is a set of bad pairings and the two players in such a pair do not want to be in the same group. Moreover, each player is free to choose which of the two groups to be in. We can model this by a graph $G = (V, E)$ where each player $i$ is a vertex. There is an edge $(i, j)$ if $i$ and $j$ form a bad pair. The private objective of player $i$ is to maximize the number of its neighbors that are in the other group.

Provide a formal characterization of the strategy profiles that are pure Nash equilibrium of this game. Analyze the complexity of the problems related to pure Nash equilibria for this family of games.}
\subsection*{Solution}
In this game, all players have de same set of actions $A = \{0,1\}$, where 0 and 1 are the two available groups which players can be partitioned in. Hence, a strategy profile for the game is $s \in \{0,1\}^n$. With this, we can define the utility of as:
\[
    u_i(s) = |\{j \in V \suchthat (i,j) \in E, s_j \neq s_i\}|
\]

In order to maximize the utility, we have to maximize the edges between both groups. So, the formal definition of the characterization of the strategy profiles that are pure Nash equilibria is the following:
\[
    \{\forall i \in V \suchthat \sum_{j \in V} u_j(s) \geq \sum_{j \in V} u_j(s_{-i}\cup \{1-s_i\})\}
\]
Notice that maximize the edges between both groups is exactly the \textit{MaxCut} problem. Because of this, we know that:
\begin{itemize}
    \item \textit{IsPNE} $\in P$ because \textit{MaxCut} is a NP-complete problem, which means that can be checked in polynomial time.
    \item You can decide \textit{EPN} in constant time, since it always exits a minimal cut for a game.
    \item Computing a \textit{MaxCut} is the same than computing a \textit{PNE}.
    \item The utility and neighborhood of a solution can be calculated and searched in polynomial time, since \textit{MaxCut} is known to be PLS-complete.
\end{itemize}

\section*{Problem 7}
{\color{statement}
\subsection*{Statement}
The Max 2SAT game is defined by a weighted 2-CNF formula on $n$ variables. In a weighted formula each clause has a weight. The game has $n$ players. Player $i$ controls the $i$-th variable and can decide the value assigned to this variable. A strategy profile is a truth assignment $x \in \{0, 1\}^n$ . Player $i$ gets $1/3$ of the weight of the clauses that are satisfied due to its bit selection.

Provide a formal characterization of the strategy profiles that are pure Nash equilibrium of the Max 2SAT game. Analyze the complexity of the problems related to pure Nash equilibria for this family of games.}

\subsection*{Solution}
Let $F$ be the 2-CNF formula, which is composed by a set of $K$ clauses $c_{i,j,k}$, where $i$ and $j$ represents the position of the variables in $x$ that forms the clause (with $1 \leq i,j \leq n$), and $k$ is the identifier of the clause (with $1 \leq k \leq K$). I define the following function to ease the explanation of the problem:
\[\text{clauses}(F,i) = \{\forall x_j \in x,  1 \leq k \leq K \suchthat c_{i,j,k} \in F\}\]
Consider also that eval$(c_{i,j,k},x)$ gives the resultant boolean value of the clause according to the strategy profile $x$. I define the set of actions for all players as $A = \{0,1\}$, and utility as:
\[
    u_i(x) = \frac{1}{3}\cdot\sum_{c_{i,j,k} \in \text{clauses}(F,i)} w_k\cdot \text{eval}(c_{i,j,k},x) \cdot (1-\text{eval}(c_{i,j,k},x_{-i}\cup \{1-x_i\}))
\]

The formal definition of the characterization of the strategy profiles that are pure Nash equilibria is the following:
\[
    \{\forall x_i \in x \suchthat \sum_{x_j\in x}u_j(x)\geq \sum_{x_j\in x}u_j(x_{-i}\cup (1-x_i)) \}
\]

Since Max 2SAT game is PNE (given an strategy profile) is a \textit{Maximal-2-SAT} problem, we know that:
\begin{itemize}
    \item \textit{IsPN} $\in P$, since \textit{Maximal-2-SAT} can be decided in polynomial time.
    \item \textit{EPN} can be decided in constant time because \textit{Maximal-2-SAT} always exists.
    \item Computing \textit{PNE} is NP-hard because \textit{Maximal-2-SAT} is known to be NP-hard.
    \item The utility and neighborhood of a solution can be calculated and searched in polynomial time, since \textit{Maximal-2-SAT} is known to be PLS-complete.
\end{itemize}
\newpage

\section*{Problem 8}
{\color{statement}
\subsection*{Statement}
Assume that we have fixed a finite set $K$ of $k$ colors. Consider a graph $G = (V, E)$ with a labeling function $\ell: V \rightarrow 2^K$ and define an associated coloring game $\Gamma(G, \ell)$ as follows
\begin{itemize}
    \item the players are $V(G)$,
    \item the set of strategies for player $v$ is $\ell(v)$,
    \item the payoff function of player $v$ is $u_v(s) = |\{u \in N(v) \suchthat s_u = s_v \}|$.
\end{itemize}

Provide a formal characterization of the strategy profiles that are pure Nash equilibrium of the coloring game game. Analyze the complexity of the problems related to pure Nash equilibria for this family of games.}

\subsection*{Solution}
In this game, each player $v$ has the set of actions $A_v \subseteq K$. Hence, a strategy profile for the game is $s \in K^n$. The formal definition of the characterization of the strategy profiles that are pure Nash equilibria, using the utility formula defined in the statement, is the following:
\[
    \{\forall s_v \in s, \forall a_v \in A_v \suchthat \sum_{s_u\in s}u_u(s)\geq \sum_{x_u\in x}u_u(x_{-v}\cup \{a_v\}) \}
\]


%\subsection*{$\mathbf{P_{||}^{NP} \subseteq \mathbf{P_{log\text{ }n}^{NP}}}$}

% \begin{lstlisting}[escapeinside={*}{*}]
% *Definition of $M'$:*
%     *Given input $X = \{x_1,x_2,...,x_n\}$*
%     *Let $B = \{b_1,b_2,...,b_n\}$ be a boolean array*
%     *\textbf{If} $M(X) = 0$ \textbf{then}*
%         *Output 0 and halt*
%     *\textbf{For} $i = 1,2,...,n$ \textbf{do}*
%         *\textbf{If} $M(\{b_0,...,b_{i-1},0,x_{i+1},...,x_n\}) = 0$ \textbf{then}*
%             *Set $b_i$ = 0*
%         *\textbf{Else if} $M(\{b_0,...,b_{i-1},1,x_{i+1},...,x_n\}) = 0$ \textbf{then}*
%             *Set $b_i$ = 1*
%         *\textbf{Else} output 0 and halt*
%     *Check if $B$ satisfies $X$*
%     *\textbf{If} it does, output 1 and halt, \textbf{else} output 0 and halt*
% \end{lstlisting}

\end{document}