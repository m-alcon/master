\documentclass[a4paper, 11pt]{article}
\usepackage{fullpage} % changes the margin
\usepackage[english]{babel}
\usepackage[utf8]{inputenc}
\usepackage{hyperref}
\usepackage{xcolor}
\usepackage{graphicx}
\usepackage{array}
\usepackage{float}
\usepackage{longtable}
\usepackage[bottom]{footmisc}
\usepackage{cite}
\usepackage{parskip}
\usepackage{subcaption}
\usepackage{amssymb}
\usepackage{amsmath}
\usepackage{listings}
\usepackage{enumitem}

\hypersetup{
	colorlinks=true,       % false: boxed links; true: colored links
	linkcolor=blue,        % color of internal links
	citecolor=blue,        % color of links to bibliography
	filecolor=magenta,     % color of file links
	urlcolor=blue
}

%\setlength{\parindent}{0cm}
\newcommand{\code}[1]{\texttt{#1}}
\newcommand{\onemm}[0]{\hspace{1mm}}
\newcommand{\suchthat}[0]{\onemm|\onemm}
\renewcommand{\arraystretch}{1.4}
\definecolor{statement}{gray}{0.5}

\graphicspath{{img/}}

\begin{document}

\noindent
\begin{flushright}
    \large\textbf{Miguel Alcón Doganoc} \\
    Algorithms for Game Theory \\
    \today
\end{flushright}

\noindent
{\huge{\textbf{First Problem Assignment}}}

\section*{Problem 17}
{\color{statement}
\subsection*{Statement}
Consider the sending from $s$ to $t$ game. Compute (or provide bounds for) the PoA and the PoS for the following social cost/utility functions:
\begin{enumerate}[label=(\alph*)]
    \item $C(s) = \begin{cases}
        \sum_{i\in N} u_i(s) & \text{if there is a path from $s$ to $t$ in $G[s]$.} \\
        n^2 & \text{otherwise.}
    \end{cases}$
    \item $U(s) = \max_{i\in N} u_i(s)$.
    \item $U(s) = \sum_{i\in N} u_i(s)$.
\end{enumerate}

}
\subsection*{Solution}
In order to ease the explanation, I define the following functions:
\begin{itemize}
    \item $LP(G)$ = ``number of vertices that are inside the longest path of $G$''.
    \item $SP(G)$ = ``number of vertices that are inside the shortest path of $G$''.
\end{itemize}
Both situations (longest and shortest paths) leads to a NE.
\begin{itemize}
    \item Longest path: all vertices that are on the way between $s$ and $t$ (i.e., all possible vertices that can be included in the path) are included in the (longest) path. Hence, all vertices are ``happy''.
    \item Shortest path: no vertex outside the (shortest) path can be part of it with only changing its own action.
\end{itemize}
With this, we can start with the computation of the bounds.
\begin{enumerate}[label=(\alph*)]
    \item Here I use the formulas of PoA and PoS for the cost functions, which are:
        \begin{align}
            &\text{PoA}(\Gamma) = \frac{max_{s \in \text{NE}(\Gamma)} C(s)}{min_{s \in A} C(s)} & \text{PoS}(\Gamma) = \frac{min_{s \in \text{NE}(\Gamma)} C(s)}{min_{s \in A} C(s)}
        \end{align}
        I know that $min_{s \in \text{NE}(\Gamma)} C(s) = min_{s \in A} C(s) = min(SP(G), n^2)$ since the minimum path will be always a NE and the shortest possible one. If $G$ is unconnected, $SP(G) = \infty$ and $\forall s \in A$, $C(s) = n^2$.

        By the statement, I also know that if there is not a path from $s$ to $t$ in G[s], then  $max_{s \in \text{NE}(\Gamma)} C(s) = n^2$.

        If $G$ is connected
        \begin{align}
            \frac{LP(G)}{SP(G)} \leq \text{PoA}(\Gamma) \leq \frac{n^2}{SP(G)}
        \end{align}
        Moreover, the largest value of $SP(G)$ is $SP(G) = LP(G)$, and the smallest value of it is $SP(G) = 1$, therefore
        \begin{align}
            1 \leq \text{PoA}(\Gamma) \leq n^2
        \end{align}
        If $G$ is unconnected, it is possible that no path from $s$ to $t$ exists. This implies that $min_{s \in A} C(s) = max_{s \in \text{NE}(\Gamma)} C(s) = n^2$ can happen. However, it does not affect the bounds.
        \begin{align}
            \frac{n^2}{n^2} = 1 \leq \text{PoA}(\Gamma) \leq n^2
        \end{align}
        As I explained before, $min_{s \in \text{NE}(\Gamma)} C(s) = min_{s \in A} C(s) = min(SP(G), n^2)$, so PoS $= 1$.
    \item Here, and in the following point, I use the formulas of PoA and PoS for the utility functions, which are:
        \begin{align}
            &\text{PoA}(\Gamma) = \frac{max_{s \in A} U(s)}{min_{s \in \text{NE}(\Gamma)} U(s)} & \text{PoS}(\Gamma) = \frac{max_{s \in A} U(s)}{max_{s \in \text{NE}(\Gamma)} U(s)}
        \end{align}
        In this case $\forall s$, $0 \leq U(s) \leq 1$.

        If $G$ is connected, there always exists a path from $s$ to $t$, so $max_{s \in A} U(s) = 1$. It can happen that a NE exists with no path from $s$ to $t$, if there is no single action from any of the vertices that connects $s$ to $t$. In this case, $min_{s \in \text{NE}(\Gamma)} U(s) = 0$. Then, we can bound PoA as follows:
        \begin{align}
            \frac{1}{1} = 1 \leq \text{PoA} \leq \infty = \frac{1}{0}
        \end{align}
        Because there always exists a path between $s$ and $t$, we can define PoS $= 1$.

        If $G$ is unconnected, $max_{s \in A} U(s) = 1$ is not true because maybe a path from $s$ to $t$ does not exist. Then, the PoA and PoS can be equal to the indeterminate form $\frac{0}{0}$.
    \item If $G$ is connected, there always exists a path from $s$ to $t$, so $max_{s \in A} U(s) = LP(G)$. As in (b), it can happen that a NE exists with no path from $s$ to $t$, so $min_{s \in \text{NE}(\Gamma)} U(s)$ can be 0, otherwise it is equal to $SP(G)$. Then
        \begin{align}
            \frac{LP(G)}{SP(G)} \leq \text{PoA} \leq \infty = \frac{LP(G)}{0}
        \end{align}
        As in (b), if $G$ is unconnected, PoA and PoS can be equal to the indeterminate form $\frac{0}{0}$.
\end{enumerate}


\section*{Problem 28}
{\color{statement}
\subsection*{Statement}
Consider a GSP auction for $n$ players. Recall that in such an auction each bid profile $b$ defines an allocation $\pi$ mapping slots to players. We say that an allocation is reasonable if for each pair $i$, $j$ of slots
\begin{equation*}
    \frac{\alpha_j}{\alpha_i} + \frac{\gamma_{\pi(i)}v_{\pi(i)}}{\gamma_{\pi(j)}v_{\pi(j)}} \geq 1.
\end{equation*}
\begin{itemize}
    \item Prove that when $b$ is a NE, the corresponding allocation $\pi$ is reasonable.
    \item Use the previous fact to show that the price of anarchy, on pure strategies, of the GSP auction is at most 2.
\end{itemize}

}
\subsection*{Solution}

\section*{Problem 35}
{\color{statement}
\subsection*{Statement}
Consider a the following decision process which runs on a leader-follower society with $n$ members. The interaction among the society is represented by a bipartite directed graph $G = (L,F,E)$ having the following property: all the vertices in $L$ have in-degree 0 and all the vertices in $F$ have out-degree 0. The decision process is defined by two parameters $\alpha$, $0 \leq \alpha \leq 1$ and $q$, $0 \leq q \leq n$.

When the society has to reach a decision about some topic, each member takes an initial position. We model this situation as an initial decision vector $x \in \{0, 1\}^n$. Then, each $i \in F$ looks at the values $p_{i1} =|\{(j,i) \in E \suchthat x_j = 1\}|$ and $p_{i0} = |\{(j,i) \in E \suchthat x_j = 0\}|$ and reconsiders its position according to the following algorithm.

\begin{itemize}
    \item If $p_{i1} > \alpha(p_{i1}+p_{i0})$ and $p_{i0} < \alpha(p_{i1}+p_{i0})$, $x_i =1$
    \item If $p_{i0} > \alpha(p_{i1}+p_{i0})$ and $p_{i1} < \alpha(p_{i1}+p_{i0})$, $x_i =0$
\end{itemize}

Finally, the society reaches a ``yea'' (1) when $\sum^n_{i=1} x_i \geq q$, and a ``nay'' (0) otherwise.
\begin{enumerate}[label=(\alph*)]
    \item Assuming that a coalition $S$ is represented as the initial decision vector $x \in \{0,1\}^n$ defined as $x_i = 1$ iff $i \in S$, the decision system process defines a cooperative game assigning to a coalition $S$ a value in $v(S) \in \{0, 1\}$. Is this game simple?
    \item Provide a characterization of the games in the family with non-empty core.
    \item Can the Banzhaf value of player $i$ be computed in polynomial time?
\end{enumerate}
}
\subsection*{Solution}


\end{document}