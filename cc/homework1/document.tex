\documentclass[a4paper, 10pt]{article}
\usepackage{fullpage} % changes the margin
\usepackage[english]{babel}
\usepackage[utf8]{inputenc}
\usepackage{hyperref}
\usepackage{xcolor}
\usepackage{graphicx}
\usepackage{array}
\usepackage{float}
\usepackage{longtable}
\usepackage[bottom]{footmisc}
\usepackage{cite}
\usepackage{parskip}
\usepackage{subcaption}
\usepackage{minted}
\usepackage{amssymb}
\usepackage{amsmath}
\usemintedstyle{vs}

\hypersetup{
	colorlinks=true,       % false: boxed links; true: colored links
	linkcolor=blue,        % color of internal links
	citecolor=blue,        % color of links to bibliography
	filecolor=magenta,     % color of file links
	urlcolor=blue
}

%\setlength{\parindent}{0cm}
\newcommand{\code}[1]{\texttt{#1}}
\renewcommand{\arraystretch}{1.4}

\graphicspath{{img/}}


\begin{document}

\noindent
\begin{flushright}
    \large\textbf{Miguel Alcón Doganoc} \\
    Computational Complexity \\
    \today
\end{flushright}

\noindent
{\huge{\textbf{Homework 1}}}

\section{Exercise: Verifiers in L}
\section{Exercise: NP = NL?}
\section{Exercise: Comparison in logspace}

As it is said in the statement, we know that $n$ is the length of $x$, and the input tape is as following:
\[
    \square \hspace{1mm} x_1,x_2,...,x_n \# y_1,y_2,...,y_n \hspace{1mm} \square
\]
Starting from this, I describe the TM that outputs 0 if $m < n$ and 1 if $m \geq n$, with the input tape defined above. For a more understandable explanation, I have made a visual representation of both tapes (first input, second work tape) at the end of each general step. Note that the position of the head of each tape is represented as a square surrounding the symbol which is in this position. Let us start with the description.

First, we have to check if $n = 0$. If it happens, we assume that both $m$ and $n$ are 0, so $m = n$.
\begin{enumerate}
    \item Scan the actual position of the head of the input tape, which is the first position starting from the left. If it is a $\#$ symbol, output 1 and halt.    
\end{enumerate}
\[
    \square \hspace{1mm} \boxed{x_1},x_2,...,x_n \# y_1,y_2,...,y_n \hspace{1mm} \square
\]
We copy $y$ into the work tape. Because $y$ is the binary encoding of $n$, we know that it is $O (\log n)$ in space.
\begin{enumerate}\setcounter{enumi}{1}
    \item Move the input tape to the right until finding the $\#$ symbol.
    \item Move the input tape to the right until finding a 1.
    \item Copy into the work tape from the first 1 (current position) to the $\square$ symbol (end of the input tape), from left to right.
\end{enumerate}
\[
    \square \hspace{1mm} x_1,x_2,...,x_n \# y_1,y_2,...,y_n \hspace{1mm} \boxed{\square}
\]
\[
    \square \hspace{1mm} y_1,y_2,...,y_n \hspace{1mm} \boxed{\square}    
\]

Before starting to check whether $m < n$ or $m \geq n$, we have to place the head of each tape in the proper position.
\begin{enumerate}\setcounter{enumi}{4}
    \item Move the input tape to the left until finding the $\#$ symbol.
    \item Move the input and the work tape simultaneously\footnote{First move one, then the other.} to the left until finding the $\square$ symbol in the work tape.
\end{enumerate}

\[
    \square \hspace{1mm} x_1,x_2,...,\boxed{x_i},...,x_n \# y_1,y_2,...,y_n \hspace{1mm} \square
\]
\[
    \boxed{\square} \hspace{1mm} y_1,y_2,...,y_n \hspace{1mm} \square   
\]

At this point, if we find a 1 between $x_1$ and $x_i$, with $i \in \{1,2,...,n\}$ or $x_i = \square$, then $m > n$.
\begin{enumerate}\setcounter{enumi}{6}
    \item Move the input tape to the left until finding a 1 or the $\square$ symbol. If it finds a 1, then it outputs 1 and halts.
\end{enumerate}
\[
    \boxed{\square} \hspace{1mm} x_1,x_2,...,x_i,...,x_n \# y_1,y_2,...,y_n \hspace{1mm} \square
\]
\[
    \boxed{\square} \hspace{1mm} y_1,y_2,...,y_n \hspace{1mm} \square   
\]

After that, if the TM haven't halted yet, we have to move the heads to the positions like at the end of step 6.
\begin{enumerate}\setcounter{enumi}{7}
    \item Move the input tape to the right until finding the $\#$ symbol.
    \item Move the working tape to the right until finding the $\square$ symbol.
    \item Repeat step 6.
\end{enumerate}

\[
    \square \hspace{1mm} x_1,x_2,...,\boxed{x_i},...,x_n \# y_1,y_2,...,y_n \hspace{1mm} \square
\]
\[
    \boxed{\square} \hspace{1mm} y_1,y_2,...,y_n \hspace{1mm} \square   
\]

Now, we continue checking whether $m < n$ or $m \geq n$, but this time comparing $x$ with $y$. 
\begin{enumerate}\setcounter{enumi}{10}
    \item Scan both tapes and, if the TM does not halt, move the head one position to the right. \begin{enumerate}
        \item If the value of the header of the input tape is 1, and the one of the work tape is 0, then output 1 and halt.
        \item Else if the value of the header of the input tape is 0, and the one of the work tape is 1, output 0 and halt.
        \item Else if the value of both heads are the $\square$ symbol (end of both tapes) output 1 and halt.
        \item Otherwise repeat step 11.
    \end{enumerate}
\end{enumerate}

\[
    \square \hspace{1mm} x_1,x_2,...,\boxed{x_i},...,x_n \# y_1,y_2,...,y_n \hspace{1mm} \square
\]
\[
    \square \hspace{1mm} y_1,y_2,...,\boxed{y_i},...,y_n \hspace{1mm} \square   
\]

\section{Exercise: Composition of logspace computable functions}
Consider the function $g':\Sigma^* \times \mathbb{N} \rightarrow \Sigma \cup \{\#\}$ that, given a string $x$ and an integer $i \in \mathbb{N}$, returns the $i$-th symbol of $g(x)$, or $\#$ if $i > |g(x)|$, where $\#$ is a symbol not in $\Sigma$. 

[\textit{Function $g'$ is computable by deterministic TMs in logarithmic space}]: Because $g(x)$ is computable by deterministic TMs in logarithmic space, then $g'(x,i)$ can compute just 1 element of the output with the same conditions. Using the same TM as $g(x)$, we can define $g'(x,i)$ as:
\begin{enumerate}
    \item Compute all the outputs of $g(x)$ until reaching the $i$-th one, without writing on the output tape.
    \item Compute the $i$-th element, output its value and halt.
\end{enumerate}

[\textit{Using $g'$ to compute $f \circ g$}]: We can compute $(f \circ g)(x)$ applying $f$ to $x$, but using $g'(x,i)$ as an input tape, i.e. as a process over $x$, launched before the scanning of the desired input, which is $g(x)$. We can map the operations of a conventional tape with the following operations of $g'(x,i)$, with $i \in \mathbb{N}$:
\begin{itemize}
    \item Moving to the right represents adding 1 to $i$.
    \item Moving to the left represents subtracting 1 to $i$.
    \item Scanning the symbol of the head represents applying $g'(x,i)$.
\end{itemize}
Notice that the bounds of this `input tape' are defined by $i \in \mathbb{N}$, and by $i > |g(x)| \Leftrightarrow$ output $\#$. Furthermore, the starting position of the input tape's head should be $i = 0$. 

Because $f(x)$ is computable by deterministic TMs in logarithmic space, and each call to $g'$ re-uses the same space (logarithmic), we can conclude then that  $(f \circ g)(x)$ is also computable by deterministic TMs in logarithmic space.

%\input{sections/end.tex}

\newpage
\bibliographystyle{unsrt}
\bibliography{sections/cites}

\end{document}
    