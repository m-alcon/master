\documentclass[letterpaper,10pt]{article}
\usepackage[utf8]{inputenc}
\usepackage{fullpage}
\usepackage[english]{babel}
\usepackage{amsmath}
\usepackage{caption}
\usepackage{subcaption}
\usepackage{graphicx}
\usepackage{amssymb}

\renewcommand{\arraystretch}{1.4}
\setlength{\parindent}{0cm}

\begin{document}

\noindent
\begin{flushright}
    \large\textbf{Miguel Alcón Doganoc} \\
    Combinatorial Problem Solving \\
    May 7, 2019
\end{flushright}

\newcommand{\code}[1]{\texttt{#1}}

\noindent
{\huge{\textbf{Logic Synthesis}}}

\section{Description of the problem}
In this project, our goal is to solve the \textit{NOR Logic Synthesis Problem}
(NLSP): given a specification of a Boolean function $f(x_1,...,x_n)$ in the form of a truth table, find a NOR-circuit satisfying the specification that minimizes depth (and, in case of a tie in depth, with minimum size). An instance of NLSP consists in:
\begin{itemize}
    \item $\mathbf{n} := $ ``Number of input signals''
    \item $\mathbf{y_t} := $ ``Desired output signal, described by row $t$ in the truth table'', where $t \in \{0,1,...,2^n-1\}$  
\end{itemize}

\section{Decision variables}
Given the number of input signals $n$, the depth $d$, and the truth table of the logical circuit, I defined the following variables:
\begin{itemize}
    \item $\mathbf{Z_{i,j}}:=$ ``The node ($i$,$j$) contains a constant zero'', where
    \begin{itemize}
        \item $0 \leq i \leq d$
        \item $0 \leq j < 2^i$
    \end{itemize}
    \item $\mathbf{N_{i,j}}:=$ ``The node ($i$,$j$) contains a NOR gate'', where
    \begin{itemize}
        \item $0 \leq i \leq d$
        \item $0 \leq j < 2^i$
    \end{itemize}
    \item $\mathbf{I_{i,j,k}}:=$ ``The node ($i$,$j$) contains the input $k$'', where
    \begin{itemize}
        \item $0 \leq i \leq d$
        \item $0 \leq j < 2^i$
        \item $1 \leq k \leq n$
    \end{itemize}
    \item $\mathbf{B_{i,j}^{(t)}}:=$ ``Boolean value of the node ($i,j$) for the row $t$ of the truth table'', where
    \begin{itemize}
        \item $0 \leq i \leq d$
        \item $0 \leq j < 2^i$
        \item $0 \leq t < 2^n$
    \end{itemize}
\end{itemize}

For example, for a NOR-circuit that implements the functionality of an AND gate (see figure \ref{fig:original}), with $n = d = 2$, one possible solution (variable assignation) is shown in figure \ref{fig:variables}.
\begin{figure}[hbtp]
    \centering
    \begin{subfigure}[b]{0.45\textwidth}
        \raisebox{18mm}{\begin{tabular}{c c | c | c}
            $x_1$ & $x_2$ & $y$ & $t$\\ \hline
            0 & 0 & 0 & 0 \\
            0 & 1 & 0 & 1 \\
            1 & 0 & 0 & 2 \\
            1 & 1 & 1 & 3 \\
        \end{tabular}}
    \end{subfigure}\hspace{-0.2\textwidth}
    \begin{subfigure}[b]{0.45\textwidth}
        \includegraphics[width=\textwidth]{circuit.pdf}
    \end{subfigure}
    \caption{Truth table of $y$ = AND($x_1,x_2$) and NOR-circuit implementing it.}
    \label{fig:original}
\end{figure}

\begin{figure}[hbtp]
    \centering
    \begin{subfigure}[b]{0.30\textwidth}
        \includegraphics[width=\textwidth]{i.pdf}
        \caption{$i,j$}
        \label{subfig:i}
    \end{subfigure}
    \hspace{0.01\textwidth}
    \begin{subfigure}[b]{0.30\textwidth}
        \includegraphics[width=\textwidth]{Z.pdf}
        \caption{$Z_{i,j}$}
        \label{subfig:zi}
    \end{subfigure}
    \hspace{0.01\textwidth}
    \begin{subfigure}[b]{0.30\textwidth}
        \includegraphics[width=\textwidth]{N.pdf}
        \caption{$N_{i,j}$}
        \label{subfig:ni}
    \end{subfigure}
    \begin{subfigure}[b]{0.30\textwidth}
        \includegraphics[width=\textwidth]{I1.pdf}
        \caption{$I_{i,j,1}$}
        \label{subfig:i1}
    \end{subfigure}
    \hspace{0.01\textwidth}
    \begin{subfigure}[b]{0.30\textwidth}
        \includegraphics[width=\textwidth]{I2.pdf}
        \caption{$I_{i,j,2}$}
        \label{subfig:i2}
    \end{subfigure}
    \hspace{0.01\textwidth}
    \begin{subfigure}[b]{0.30\textwidth}
        \includegraphics[width=\textwidth]{bit.pdf}
        \caption{$B_{i,j}^{(0)}$}
        \label{subfig:bit}
    \end{subfigure}
    \caption{Visual representation of ($i,j$) and the variables.}
    \label{fig:variables}
\end{figure}

\section{Constraints}
In order to simplify the definition of the constraints, I define three functions. Given the variable $v_{i,j}$, with $v_{i,j} \in \{Z_{i,j}; N_{i,j}; I_{i,j,k}; B_{i,j}^{(t)}\}$,
\begin{itemize}
    \item \textbf{left($v_{i,j}$)} := ``Variable corresponding to the one on the left of $v_{i,j}$'' $ = v_{i+1,2\times j}$.
    \item \textbf{right($v_{i,j}$)} := ``Variable corresponding to the one on the right of $v_{i,j}$'' $ = v_{i+1,2\times j+1}$.
    \item \textbf{bit($k,t$)} := ``Boolean value of $x_k$ in the $t$-th row of the truth table'' = ``Value of the position $k$ of the binary representation of $t$ (i.e. $t_k \in \{t_1 t_2 ... t_n$\})''.
\end{itemize}

So, the constraints are the following:
\begin{itemize}
    \item The output of the circuit is equal to the desired value for each row $t$ of the truth table.
    \begin{align*}
        B_{0,0}^{(t)} = y_t \\
        \forall t < 2^n
    \end{align*}
    \item NOR gates are not allowed on the leaves of the circuit.
    \begin{align*}
        N_{d,j} = 0 \\
        \forall j < 2^d 
    \end{align*}
    \item Force children (if any) of each node to be 0 if the node is not a NOR gate.
    \begin{align*}
        \mathbf{left}(Z_{i,j}) + \mathbf{right}(Z_{i,j}) \geq (1-N_{i,j})\cdot 2 \\
        \forall i < d\text{ }\forall j < 2^i
    \end{align*}
    \item Force non-symmetry of NOR gates' children.
    \begin{itemize}
        \item Allow an input on the left only if there is an input on the right.
            \begin{align*}
                \mathbf{left}(I_{i,j,k}) + \mathbf{right}(Z_{i,j}) + \mathbf{right}(N_{i,j}) \leq 1 \\
                \forall i < d\text{ }\forall j < 2^i
            \end{align*}
        \item Allow a constant zero on the left only if there is not a NOR gate on the right.
            \begin{align*}
                \mathbf{left}(Z_{i,j}) + \mathbf{right}(N_{i,j}) \leq 1  \\
                \forall i < d\text{ }\forall j < 2^i
            \end{align*}
        \item Do not allow the same input on both sides.
            \begin{align*}
                \mathbf{left}(I_{i,j,k}) + \mathbf{right}(I_{i,j,k}) \leq 1 \\
                \forall i < d\text{ }\forall j < 2^i\text{ }\forall k \leq n
            \end{align*}
        \item Inputs on the left must be smaller than the ones on the right.
            \begin{align*}
                \mathbf{left}(I_{i,j,k}) + \mathbf{right}(I_{i,j,l}) \leq 1 \\
                \forall i < d\text{ }\forall j < 2^i\text{ }\forall k,l < n\text{ with }k < l
            \end{align*}
    \end{itemize}
    \item Link each NOR gate with its corresponding value, which is the NOR operation between both children.
    \begin{align*}
        (1 - \mathbf{right}(B_{i,j}^{(t)}) - \mathbf{left}(B_{i,j}^{(t)})) - 2\cdot B_{i,j}^{(t)} \leq 1-N_{i,j} \\
   (1 - \mathbf{right}(B_{i,j}^{(t)}) - \mathbf{left}(B_{i,j}^{(t)})) - 2\cdot B_{i,j}^{(t)} \geq 2 \cdot N_{i,j} - 3 \\
        \forall t < 2^n\text{ }\forall i < d\text{ }\forall j < 2^i
    \end{align*}
    \item Link each constant 0 signal with `false'.
    \begin{align*}
        1 - Z_{i,j} \geq B_{i,j}^{(t)} \\
        \forall t < 2^n\text{ }\forall i \leq d\text{ }\forall j < 2^i
    \end{align*}
    \item Link each input signal that has value 1 in the truth table, with `true'.
    \begin{align*}
        1 - I_{i,j,k} \geq B_{i,j}^{(t)} - \mathbf{bit}(k,t) \\
        1 - I_{i,j,k} \geq \mathbf{bit}(k,t) - B_{i,j}^{(t)} \\
        \forall 1 \leq k \leq n\text{ }\forall t  < 2^n\text{ }\forall i \leq d\text{ }\forall j < 2^i
    \end{align*}
    \item Force each node to be only of one type.
    \begin{align*}
        Z_{i,j} + N_{i,j} + \sum_{k = 1}^n {I_{i,j,k}} = 1\\
        \forall i < d\text{ }\forall j < 2^i
    \end{align*}
\end{itemize}

We want to \textbf{minimize} the size of the circuit:
\begin{align*}
    \sum_{i=0}^{d-1} \sum_{j=0}^{2^i-1} N_{i,j} \\
\end{align*}

\end{document}